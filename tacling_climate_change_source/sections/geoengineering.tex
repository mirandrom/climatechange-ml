\documentclass[11pt]{report}
\usepackage[margin=2cm]{geometry}
\usepackage{graphicx}
\usepackage{float}
\usepackage{times}
\newcommand{\carbon}{CO\textsubscript{2}}

\newcommand{\Gap}{\texorpdfstring{\hfill}{}}
\newcommand{\Rec}{\texorpdfstring{{\small\emph{\color{blue}{\fbox{High Leverage}}}}}{}}
\newcommand{\HighRisk}{\texorpdfstring{{\small\emph{\color{orange}{\fbox{Uncertain Impact}}}}}{}}
\newcommand{\Longterm}{\texorpdfstring{{\small\emph{\color{OliveGreen}{\fbox{Long-term}}}}}{}}

\usepackage[dvipsnames]{xcolor}

\begin{document}

\section{Solar Geoengineering\texorpdfstring{\hfill\textit{by Andrew S.~Ross}}{}}
\label{sec:geoengineering}

Airships floating through the sky, spraying aerosols; robotic boats crisscrossing the ocean, firing vertical jets of spray; arrays of mirrors carefully positioned in space, micro-adjusted by remote control: these images seem like science fiction, but they are actually real proposals for solar radiation management, commonly called solar geoengineering \cite{keith2000geoengineering,shepherd2009geoengineering,irvine2016overview,keith2018science}. Solar geoengineering, much like the greenhouse gases causing climate change, shifts the balance between how much heat the Earth absorbs and how much it releases. The difference is that it is done deliberately, and in the opposite direction. The most common umbrella strategy is to make the Earth more reflective, keeping heat out, though there are also methods of helping heat escape (besides \carbon~removal, which we discuss in \textsection\ref{sec:afolu} and \textsection\ref{sec:ccs}).

Solar geoengineering generally comes with a host of potential side effects and governance challenges. Moreover, unlike \carbon~removal, it cannot simply reverse the effects of climate change (average temperatures may return to pre-industrial levels, but location-specific climates still change), and also comes with the risk of \emph{termination shock} (fast, catastrophic warming if humanity undertakes solar geoengineering but stops suddenly) \cite{parker2018risk}. Because of these and other issues, it is not within the scope of this paper to evaluate or recommend any particular technique. However, the potential for solar geoengineering to moderate some of the most catastrophic hazards of climate change is well-established \cite{irvine2019halving}, and it has received increasing attention in the wake of societal inaction on mitigation. Although \cite{keith2018science} argue that the ``hardest and most important problems raised by solar geoengineering are non-technical,'' 
there are still a number of important technical questions that machine learning may be able to help us study.

\paragraph*{Overview}\Gap\mbox{}\\
The primary candidate methods for geoengineering are marine cloud brightening \cite{jones2009climate} (making low-lying clouds more reflective), cirrus thinning \cite{storelvmo2014cirrus} (making high-flying clouds trap less heat), and stratospheric aerosol injection \cite{rasch2008overview} (which we discuss below). Other candidates (which are either less effective or harder to implement) include ``white-roof'' methods \cite{akbari2012long} and even launching sunshades into space \cite{angel2006feasibility}.

Injecting sulfate aerosols into the stratosphere is considered a leading candidate for solar geoengineering both because of its economic and technological feasibility \cite{mcclellan2012cost,smith2018production} and because of a reason that should resonate with the ML community: we have data. (This data is largely in the form of temperature observations after volcanic eruptions, which release sulfates into the stratosphere when sufficiently large \cite{robock2013studying}.) Once injected, sulfates circulate globally and remain aloft for 1 to 2 years. As a result, the process is reversible, but must also be continually maintained. Sulfates come with a well-studied risk of ozone loss \cite{eastham2018quantifying}, and they make sunlight slightly more diffuse, which can impact agriculture \cite{proctor2018estimating}.

\subsection{Understanding and improving aerosols}
\label{subsub:better-aerosols}

\paragraph*{Design}\Gap\mbox{\Longterm}\\
The effects and side-effects of aerosols in the stratosphere (or at slightly lower altitudes for cirrus thinning \cite{doi:10.1029/2018JD029815}) vary significantly with their optical and chemical properties. Although sulfates are the best understood due to volcanic eruption data, many others have been studied, including zirconium dioxide, titanium dioxide, calcite (which preserves ozone), and even synthetic diamond \cite{dykema2016improved}. However, the design space is far from fully explored. Machine learning has had recent success in predicting or even optimizing for specific chemical and material properties \cite{raccuglia2016machine,liu2017materials,gomez2018automatic,butler2018machine}. Although speculative, it is conceivable that ML could accelerate the search for aerosols that are chemically nonreactive but still reflective, cheap, and easy to keep aloft.

\paragraph*{Modeling}\Gap\mbox{}\\
One reason that sulfates have been the focus for aerosol research is that atmospheric aerosol physics is not perfectly captured by current climate models, so having natural data is important for validation. Furthermore, even if current aerosol models are correct, their best-fit parameters must still be determined (using historical data), which comes with uncertainty and computational difficulty. ML may offer tools here, both to help quantify and constrain uncertainty, and to manage computational load. As a recent example, \cite{fletcher2018quantifying} use Gaussian processes to emulate climate model outputs based on nine possible aerosol parameter settings, allowing them to establish plausible parameter ranges (and thus much better calibrated error-bars) with only 350 climate model runs instead of $>$100,000. Although this is important progress, ideally we want uncertainty-aware aerosol simulations with a fraction of the cost of one climate model run, rather than 350. ML may be able to help here too (see \textsection\ref{sec: climate prediction} for more details).

\subsection{Engineering a planetary control system\Gap \Rec \Longterm \HighRisk}
\label{subsub:planetary-control}
Efficient emulations and error-bars will be essential for what MacMartin and Kravitz \cite{macmartin2018engineering} call ``The Engineering of Climate Engineering.'' According to \cite{macmartin2018engineering}, any practical deployment of geoengineering would constitute ``one of the most critical engineering design and control challenges ever considered: making real-time decisions for a highly uncertain and nonlinear dynamic system with many input variables, many measurements, and a vast number of internal degrees of freedom, the dynamics of which span a wide range of timescales.'' Bayesian and neural network-based approaches could facilitate the fast, uncertainty-aware nonlinear system identification this challenge might require. Additionally, there has been recent progress in reinforcement learning for control \cite{moe2018machine,boczar2018finite,amos2018differentiable}, which could be useful for fine-tuning geoengineering interventions such as deciding where and when to release aerosols. For an initial attempt at analyzing stratospheric aerosol injection as a reinforcement learning problem (using a neural network climate model emulator), see \cite{de2019stratospheric}.

\subsection{Modeling impacts \Gap \Longterm}
\label{subsub:impact-models}
Of course, optimizing interventions requires defining objectives, and the choices here are far from clear. Although it is possible to stabilize global mean temperature and even regional temperatures through geoengineering, it is most likely impossible to preserve all relevant climate characteristics in all locations. Furthermore, climate model outputs do not tell the full story; ultimately, the goal of climate engineering is to minimize harm to people, ecosystems, and society. It is therefore essential to develop robust tools for estimating the extent and distribution of these potential harms.  There has been some recent work in applying ML to assess the impacts of geoengineering. For example, \cite{di2016assessing} use deep neural networks to estimate the effects of aerosols on human health, while \cite{crane2018using} use them to estimate the effects of solar geoengineering on agriculture. References \cite{burke2015global,diffenbaugh2019global} use relatively simple local and polynomial regression techniques but applied to extensive empirical data to estimate the past and future effects of temperature change on economic production. More generally, the field of \emph{Integrated Assessment Modeling}  \cite{kelly1999integrated,weyant2017some} aims to map the outputs of a climate model to societal impacts; for a general discussion of potential opportunities for applying ML to IAMs, see \S\ref{sec:decisionmaking}.

\subsection{Discussion}

Any consideration of solar geoengineering raises many moral questions. 
It may help certain regions at the expense of others, introduce risks like termination shock, and serve as a ``moral hazard'': widespread awareness of its very possibility may undermine mainstream efforts to cut emissions \cite{lin2013does}. Because of these issues, there has been significant debate about whether it is ethically responsible to research this topic \cite{preston2013ethics,keith2017toward}. However, although it creates new risks, solar geoengineering could actually be a moderating force against the terrifying uncertainties climate change already introduces \cite{macmartin2015solar,irvine2019halving}, and ultimately many environmental groups and governmental bodies have come down on the side of supporting further research.\footnote{ \url{https://www.edf.org/climate/our-position-geoengineering}}\footnote{\url{https://www.nrdc.org/media/2015/150210}}\footnote{\url{https://www.ucsusa.org/sites/default/files/attach/2019/gw-position-Solar-Geoengineering-022019.pdf}}
In this section, we have attempted to outline some of the technical challenges in implementing and evaluating solar geoengineering. We hope the ML community can help geoengineering researchers tackle these challenges.
\end{document}
